\documentclass[11pt]{scrartcl}
\usepackage[sexy]{evan}
\usepackage{amsmath}
\makeatletter
\@addtoreset{equation}{section} % reset equation when next section
\makeatother
\renewcommand{\theequation}{\thesection.\arabic{equation}} % \theequation is section.equation
\newcommand\numberthis{\refstepcounter{equation}\tag{\theequation}} % increase equation and write the format defined by theequation(section.equation)
\ihead{\footnotesize\textbf\thetitle} \ohead{\footnotesize\href{https://github.com/mayankxor}{\ttfamily Github}, last compiled \today}
\title{Computational Methods, Unit 3}
\author{Mayank Rajput}
\date{October $4^{th},\ 2025$}
\begin{document}
\maketitle
\begin{abstract}
  System of Linear Algebraic Equations: Existence of solution, Gauss elimination method and its computational effort, concept of Pivoting, Gauss Jordan method and its computational effort, Triangular Matrix factorization methods: Dolittle algorithm, Crout's Algorithm, Cholesky method, Eigen value problem: Power method, Approximation by Spline Function: First-Degree and second degree Splines, Natural Cubic Splines, B Splines, Interpolation and Approximation
\end{abstract}

\tableofcontents
\newpage

\addtocounter{section}{-1}
\section{Syllabus}
\section{System of Linear Algebraic Equations}
A system of $m$ equations with $n$ unknowns is has the form:
\begin{align*}
  A_{1,1}X_1+A_{1,2}X_2+A_{1,3}X_3+\cdots+A_{1,n}X_n&=B_1\\
  A_{2,1}X_1+A_{2,2}X_2+A_{2,3}X_3+\cdots+A_{2,n}X_n&=B_2\\
  A_{3,1}X_1+A_{3,2}X_2+A_{3,3}X_3+\cdots+A_{3,n}X_n&=B_3\\
  \cdots\cdots\cdots\cdots\cdots\cdots\cdots\cdots\cdot\cdots\cdots\cdots\cdots\cdots&\cdots\cdots\\
  A_{m,1}X_1+A_{m,2}X_2+A_{m,3}X_3+\cdots+A_{m,n}X_n&=B_m
\end{align*}
where $\{A_{i,j}\}_{(1,1)}^{(m,n)}$ and $\{B\}_{1}^m$ are known constants and $\{X\}_1^n$ are unknown variables such that
all these equations are linearly independent of each other.
Two equations are called ``linearly independent" if one of them \textbf{can't} be obtained by the other.
For example, $3x+6y=3$ is linearly dependent with $x+2y=1$ because one can be obtained by other(factor out $3$).
But $3x+6y=5$ and $x+2y=1$ are linearly independent.

More specifically,
\begin{align*}
  A_1x_1+B_2x_2+A_3x_3+\cdots+A_nx_n=p\\
  B_1x_1+B_2x_2+B_3x_3+\cdots+B_nx_n=q
\end{align*}
are linearly dependent iff
\[\frac{A_1}{B_1}=\frac{A_2}{B_2}=\frac{A_3}{B_3}=\cdots=\frac{A_n}{B_n}=\frac{p}{q}\]
Otherwise, they are linearly independent of each other.
\subsection{Existence of solution}
In a system of simultaneous equations with $n$ unknowns and $m$ linearly independent equations
\begin{itemize}
  \item{$m=n$}: There exists a \textbf{unique solution} set $\{X\}_1^n$
  \item{$m<n$}: There exists \textbf{many sets of solution} sets. This system is called ``under determined"
  \item{$m>n$}: A \textbf{solution set may or may not exist} in this case, its called over determined system
\end{itemize}
\subsection{Gauss Elimination Method}
For $n$ unknowns, the solution set is obtained in $n-1$ steps. Each step eliminates a variable
so after all the steps, only one variable is left, which constitutes a linear equation solvable by
a $3^{rd}$ grader. We then substitute everything back. The $k^{th}$ variable is eliminated in the $k^{th}$ step from all equations
with serial number $>k$.

Essentially, we need to convert the square matrix $A$ in $AX=B$ into an upper triangular matrix $A'$ in $A'X=B'$
using row operations.\\
To eliminate $X_k$ from $i^{th}$ equation$(i>k)$, we multiply $k^{th}$ equation by $-A_{i,k}/A_{k,k}$ and add it to
the $i^{th}$ equation. The $j^{th}$ coefficient of $i^{th}$ equation is obtained via
\newpage
\begin{align*}
  A_{i,j}^{(k)}&=A_{i,j}+u^{*}A_{k,j}\\
  b_i^{(k)}&=b_i+u^*b_k
\end{align*}
for $1\ge j\ge n$ where $u=-(A_{i,k}/A_{k,k})$

The simplified expression for solution set is
\[\{X\}_i=\left(b_i^{(i-1)}-\sum_{j=i+1}^{n}{A_{i,j}^{(i-1)}X_j}\right)/{A_{i,i}^{(i-1)}}\]
\subsubsection{Division by Zero}
To eliminate $X_k$ from $i^{th}$ equation, we multiply $k^{th}$ equation by $-A_{i,k}/A_{k,k}$. Sometimes, 
$A_{k,k}$ can be zero. This leads to unexpected(wrong) results.
\subsubsection{Number of operations}
In reducing a system of $m$ equations with $n$ unknowns, the number of product operations involved are
$\sum_{k=1}^{n-1}(n-k)(n-k+1)$. The number of subtraction operations are $\sum_{k=1}^{n-1}(n-k)(n-k+1)$ as well.
And the number of division operations are $n-1$\\
While in the back substitution process, the total number of operations involved are $1+2\sum_{k=1}^{n-1}n-k+1$

The total number of operations involved in this method are
\begin{align*}
  &1+2\sum_{k=1}^{n-1}\left((n-k)^2+(n-k)+(n-k+1)\right)\\
  =\ &1+4\sum_{k=1}^{n-1}\left(n^2-2nk+k^2+n-k+n-k+1\right)\\
  =\ &1+2\left(\sum_{k=1}^{n-1}k^2\right)-(4n+4)\left(\sum_{k=1}^{n-1}k\right)+(2n^2+2n+2n+2)\left(\sum_{k=1}^{n-1}1\right)\\
  =\ &1+2(n+1)^2(n-1)-2n(n-1)(n+1)+n(n-1)(2n-1)/6\\
  =\ &2n^3/3+n^2+n/3-1
\end{align*}
\end{document}
