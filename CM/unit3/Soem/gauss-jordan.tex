\documentclass[11pt]{scrartcl}
\usepackage[sexy]{evan}
\usepackage{amsmath}
\makeatletter
\@addtoreset{equation}{section}
\makeatother
\renewcommand{\theequation}{\thesection.\arabic{equation}}
\newcommand\numberthis{\refstepcounter{equation}\tag{\theequation}}
\author{Mayank Rajput}
\title{Gauss Jordan Method}
\begin{document}
\maketitle
This method is used to find the inverse of a non singular matrix.
We know that every non singular matrix $A$ has a unique inverse
$A^{-1}$ such that 
\[A^{-1}=\frac{\text{adj}A}{|A|}\] 
where $\text{adj}A$ is the adjoint of $A$ defined as 
\[\text{adj}A=\begin{bmatrix}
  A_{11}&A_{21}&A_{31}&\cdots&A_{n1}\\
  A_{12}&A_{22}&A_{32}&\cdots&A_{n2}\\
  \vdots&\vdots&\vdots&\ddots&\vdots\\
  A_{1n}&A_{2n}&A_{3n}&\cdots&A_{nn}
\end{bmatrix}\]
where $A_{ij}$ is the cofactor of $a_{ij}$ in $|A|$

In Gauss Jordan Method, the matrix $A$ is augmented with a unit matrix of same dimension.
So the augmented matrix for $A_{n\times n}$ will be $\left[A:I\right]_{n\times 2n}$
\[\left[A:I\right]=\begin{bmatrix}
  a_{11}&a_{21}&a_{31}&\vdots&1&0&\cdots&0\\
  a_{21}&a_{22}&a_{31}&\vdots&0&1&\cdots&0\\
  \vdots&\vdots&\vdots&\vdots&\vdots&\vdots&\vdots&\vdots\\
  a{n1}&a_{n2}&a_{n3}&\vdots&0&0&\cdots&1
\end{bmatrix}\]
The way to find inverse of $A$ is to convert the left half of augmented matrix to
an upper triangular matrix, and then to an identity matrix.
At the end $\left[A:I\right]$ is converted to $\left[I:A^{-1}\right]$ and inverse is
found at the right side of the augmented matrix.

For this, we move the largest element(by magnitude) of first column to the first row,
using row interchange operation. Then move the largest second column to second row and so on.

\end{document}
