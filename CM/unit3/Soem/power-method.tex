\documentclass[11pt]{scrartcl}
\usepackage[sexy]{evan}
\usepackage{amsmath}
\makeatletter
\@addtoreset{equation}{section}
\makeatother
\renewcommand{\theequation}{\thesection.\arabic{equation}}
\newcommand\numberthis{\refstepcounter{equation}\tag{\theequation}}
\author{Mayank Rajput}
\title{Power Method}
\begin{document}
\maketitle
This method is used to find the largest eigenvalue and corresponding eigenvector
for a matrix $A_{n\times n}$, often called the first eigenvalue.

Let $\left\{\lambda_i\right\}_{i=1}^{n}$ be the eigenvalues of $A$
with the corresponding eigenvectors being $\left\{X_i\right\}_{i=1}^n$
This method is only applicable if all the eigenvectors are linearly independent of each other.
So that
\[X=\sum_{i=1}^n{c_iX_i}\]
We take an arbitrary initial vector called $X^{(0)}_{n\times1}$ and find $AX^{(0)}$. Then
factor out the largest number from the resultant matrix, call it $\lambda^{(0)}$.
\[AX^{(0)}=\lambda^{(0)}X^{(1)}\]
Now find $AX^{(1)}$ and factor out the largest number called $\lambda^{(1)}$ to leave $X^{(2)}$.

Generally:
\[AX^{(k)}=\lambda^{(k)}X^{(k+1)}\]
We stop when $X^{(r-1)}\approx X^{(r)}$. Then $\lambda^{(r)}$ is the largest eigenvalue and
$X^{(r)}$ the corresponding eigenvector
\end{document}
