\documentclass[11pt]{scrartcl}
\usepackage[sexy]{evan}
\usepackage{amsmath}
\makeatletter
\@addtoreset{equation}{section}
\makeatother
\ihead{\footnotesize\textbf\thetitle} \ohead{\footnotesize\href{https://github.com/mayankxor/LaTeX-this-and-that/blob/main/CM/unit3/Soem/splines.tex}{\ttfamily Source Code}, last compiled \today}
\renewcommand{\theequation}{\thesection.\arabic{equation}}
\newcommand\numberthis{\refstepcounter{equation}\tag{\theequation}}
\author{Mayank Rajput}
\title{Splines}
\begin{document}
\maketitle
Spline interpolation is very powerful and widely used method for interpolation
and has many applications. It interpolates a function between given set of points
by means of piecewise smooth polynomials.
\section{Linear Spline}
for given values $y(x_i)=y_i$ for $i=0,1,2,\cdots,n$.
We create linear functions for every subinterval.
This is literally a graph made by joining the points with straight lines.
The function is continous but not the slope or curvature
\section{Quadratic spline}
If the given set of values $y(x_i)=y_i$ for $i = 1,2,3,\cdots,n$
has $p(x)$ as the quadratic spline function, then $p(x)$ must obey the following:
\begin{enumerate}
  \item
    $y_i=p(x_i)\forall i$
  \item
    $p(x)$ is a quadratic for every subinterval, except possibly the first or last subinterval
  \item
    $p(x)$ and $p'(x)$ are continuous in $(a,b)$
\end{enumerate}
\subsection{Disadvantage}
\begin{enumerate}
  \item There might be a straight line connecting the first(or last) subinterval.
  \item Curvature is not guaranteed to be continuous, Hence the graph may not be pleasant to the eyes ): 
\end{enumerate}
\section{Cubic spline}
If the given set of values $y(x_i)=y_i$ for $i = 1,2,3,\cdots,n$
has $p(x)$ as the cubic spline function, then $p(x)$ must obey the following:
\begin{enumerate}
  \item
    $y_i=p(x_i)\forall i$
  \item
    $p(x)$ is a cubic polynomial for every subinterval
  \item
    $p(x)$, $p'(x)$ and $p''(x)$ are continuous in $(a,b)$
\end{enumerate}
$p''(x)$ is also written as $M_x$
\subsection{Natural Cubic Spline}
if $p''(x_0)=p''(x)=0$, then the subic spline $p(x)$ is said to be a natural cubic spline.
For natural cubic splines, we have an explicit formula

Say the value set $y(x_i)=y_i$ is to be interpolated for $i=0,1,2,\cdots,n$ using a natural cubic spline.
We will have $n$ subintervals, and each subinterval will have a cubic function associated to it.
We define the spline $p_i(x)$ for the interval $[x_{i-1},x_i]$ for $i=1,2,3,\cdots,n$.
then
\begin{align*}p_i(x)=&\frac{1}{6h}\left[(x_i-x)^3M_{i-1}+(x-x_{i-1})M_i\right]\\&+\frac{1}{h}(x_i-x)\left(y_{i-1}-\frac{h^2}{6}M_{i-1}\right)\\&+\frac{1}{h}(x-x_{i-1})\left(y_i-\frac{h^2}{6}M_i\right)\end{align*}
for $1\le i\le n$,
where \[M_{i-1}+4M_i+M_{i+1}=\frac{6}{h^2}(p_{i+1}-2p_i+p_{i-1})\] for $1\le i<n$\\
This is only applicable if the difference in given $x$ values is constant.
\end{document}
