\documentclass{book}
\usepackage[utf8]{inputenc}
\usepackage{amsfonts, amsmath, amssymb}
\usepackage{hyperref}
\usepackage[left=2cm,right=2cm,top=1cm,bottom=1.6cm]{geometry}

\author{Mayank Rajput}
\date{\today}
\title{Higher Engineering Mathematics}

\begin{document}
\maketitle
\tableofcontents
\chapter{Solution of Equations}
\section{Introduction}
The expression $f(x)=a_{0}x^{n}+a_{1}x^{n-1}+\ldots+a_{n-1}x+a_n$\\
where $\{a_i\}_{i=0}^{n}$ are constants and $a_0\neq0$ and $n\in\mathbb{Z}^{+}$ , is called a $polynomial\ in\ x\ of\ degree\ n$. The polynomial $f(x)=0$ is called an $algebraic\ equation\ of\ degree\ n$. If $f(x)$ contains some other function like trigonometric, logarithmic, exponential, etc. then $f(x)=0$ is said to be a $transcendental\ equation$

\begin{equation}
\text{The value of } x \text{ for which}f(x)=0, 	\label{eq:1.1}
\end{equation}
is called its $root\ or\ solution$. Geometrically, a root of $(\ref{eq:1.1})$ is that value of $x$ where the graph of $y=f(x)$ crosses the x-axis. We often come across problems in deflection of beams, electrical circuit and mechanical vibration which depend upon the $solution$ of these equations. Hence, they are of great importance in the field of Applied as well as Pure Mathematics.

\section{General Properties}
\begin{enumerate}
\item If $\alpha$ is a $root\ of\ the\ polynomial\ equation\ f(x)=0$, then the polynomial $f(x)$ is $exactly\ divisible\ by \ x-\alpha$ and conversely.\\
\textbf{For example:} $x=3$ is a solution of $x^{4}-6x^{2}-8x-3=0$ since it satisfies the equation.\\
$\therefore$ $(x-3)$ is a factor of $x^{4}-6x^{2}-8x-3=0$

\item Every $algebraic\ equation\ of\ degree\ n\ has\ n\ roots\ (real\ or\ imaginary)$.\\
Conversely, if $\{\alpha_{i}\}_{i=1}^{n}=\{\alpha_{1}, \alpha_{2}, \alpha_{3}\ldots\alpha_{n}\}$ are the roots of $nth\ degree\ polynomial\ {f(x)=0}$, then $$f(x)=A\prod_{i=1}^{n}(x-\alpha_{i})=A(x-\alpha_{1})(x-\alpha_2)(x-\alpha_3)\cdots(x-\alpha_{n})$$where $A$ is a constant.

$\textbf{Obs1. }\boxed{\text{a polynomial of }n\text{ degree has more than }n\text{, then it must be identically zero}}$

\item\textbf{Intermediate Value Property}: If $f(a)$ and $f(b)$ have different signs, then the equation $f(x)=0$ must have an $odd\ number\ of\ roots$ between $x=a$ and $x=b$.

Similarly, if $f(a)$ and $f(b)$ have different signs, then the equation $f(x)=0$ must have an $even\ number\ of\ roots$ between $x=a$ and $x=b$
\end{enumerate}
\end{document}